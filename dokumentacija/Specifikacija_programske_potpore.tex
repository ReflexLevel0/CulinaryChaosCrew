\chapter{Specifikacija programske potpore}
		
	\section{Funkcionalni zahtjevi}

			\noindent \textbf{Dionici:}
			
			\begin{packed_enum}
				
				\item Korisnik
				\item Administrator				
				\item Razvojni tim
				\item Vanjski sustav za komunikaciju
				
			\end{packed_enum}
			
			\noindent \textbf{Aktori i njihovi funkcionalni zahtjevi:}
			
			
			\begin{packed_enum}
				\item  \underbar{Neregistrirani korisnik (inicijator) može: }
				\begin{packed_enum}
					\item vidjeti listu recepata svih korisnika
					\item pretražiti recepte po naslovu i korisniku
					\begin{packed_enum}
						\item filtrirati listu prema kategoriji, vrsti kuhinje i sastojcima
					\end{packed_enum}
					\item pregledati detalje zasebnih recepata
					\item stvoriti novi korisnički račun
				\end{packed_enum}
			
				\item  \underbar{Registrirani korisnik (inicijator) može:}
				\begin{packed_enum}
					\item ulogirati se pomoću vlastitog korisničkog računa
					\item objaviti nove recepte
					\begin{packed_enum}
						\item odrediti kategoriju recepta
						\item dodati posebne značajke i tagove na recept
						\item priložiti slike ili videozapise receptu
					\end{packed_enum}
					\item pregledati listu vlastitih recepata
					\item obrisati vlastiti recept
					\item komunicirati s drugim registriranim korisnicima
					\begin{packed_enum}
						\item razmjenjivati poruke i pozive
						\item postaviti termine otvorene za komunikaciju
					\end{packed_enum}
					\item označiti, komentirati i spremiti recepte
					\item pratiti omiljene autore i primati obavijesti o njihovim novim objavama
					\item promijeniti detalje korisničkog računa:
					\begin{packed_enum}
						\item korisničko ime
						\item lozinku
					\end{packed_enum}
					\item deaktivirati vlastiti korisnički račun
				\end{packed_enum}
			
				\item\underbar{Administrator (inicijator) može:}
				\begin{packed_enum}
					\item pregledati objavljene recepte
					\item promijeniti korisničke podatke registriranih korisnika
					\item promijeniti podatke o receptu
					\item izbrisati korisnički račun
					\item izbrisati recept
				\end{packed_enum}

				\item\underbar{Online-chat platforma (sudionik) može:}
				\begin{packed_enum}
					\item slati i primati poruke između registriranih korisnika
					\item pohranjivati poruke
					\item pratiti i prikazivati dostupnost registriranih korisnika
					\item uspostaviti videopoziv među registriranim korisnicima
				\end{packed_enum}
			
			\end{packed_enum}
			
			\eject 
			
			
				
			\subsection{Obrasci uporabe}

					\noindent \underbar{\textbf{UC1 - stvaranje korisničkog računa}}
					\begin{packed_item}
						\item \textbf{Glavni sudionik: neregistrirani korisnik}
						\item  \textbf{Cilj: stvaranje novog korisničkog računa}
						\item  \textbf{Sudionici: -}
						\item  \textbf{Preduvjet: -}
						
						\item  \textbf{Opis osnovnog tijeka:}
						\item[] \begin{packed_enum}
							\item Korisnik klikne na gumb za prijavu
							\item Korisnik upiše  željeno korisničko ime i lozinku
							\item Korisnik klikne na gumb za registraciju
							\item Šalje se zahtjev sustavu
							\item Sustav zapisuje novog korisnika u bazu podataka
							\item Ako je registracija uspješna web stranica se preusmjeri na homepage, ali sa korisnikom prijavljenim u sustav
						\end{packed_enum}
						
						\item  \textbf{Opis mogućih odstupanja:}
						\item[] \begin{packed_item}							
							\item[5.a] Korisničko ime je zauzeto (već postoji)
							\begin{packed_enum}
								\item Prikazuje se poruka "Korisničko ime zauzeto!"
							\end{packed_enum}
							\item[5.b] Lozinka je slaba (sadrži manje od 6 znakova i/ili sadrži samo slova)
							\begin{packed_enum}
								\item Prikazuje se poruka "Lozinka mora imati najmanje 6 znakova te barem jedan broj ili specijalan znak!"
							\end{packed_enum}
							\item[5.c] Unesena adresa e-pošte nije važeća
							\begin{packed_enum}
								\item Prikazuje se poruka "Adresa e-pošte ne postoji ili nije važeća!"
							\end{packed_enum}
						\end{packed_item}
					\end{packed_item}


				
					\noindent \underbar{\textbf{UC2 - prijava u korisnički račun}}
					\begin{packed_item}
						\item \textbf{Glavni sudionik: registrirani korisnik}
						\item  \textbf{Cilj: prijava korisnika u vlastiti račun}
						\item  \textbf{Sudionici: -}
						\item  \textbf{Preduvjet: -}
						
						\item  \textbf{Opis osnovnog tijeka:}
						\item[] \begin{packed_enum}
							\item Korisnik klikne na gumb za prijavu
							\item Korisnik unosi korisničko ime i lozinku
							\item Korisnik klikne na gumb za prijavu
							\item Šalje se zahtjev sa korisničkim podacima sustavu
							\item Sustav provjerava točnost podataka u bazi podataka
							\item U slučaju da je korisnik unio točne podatke, web stranica se preusmjeri na homepage sa korisnikom koji je sada prijavljen
						\end{packed_enum}
						
						\item  \textbf{Opis mogućih odstupanja:}
						\item[] \begin{packed_item}
							\item[5.a] Korisničko ime ne postoji u bazi podataka
							\begin{packed_enum}
								\item Prikazuje se poruka "Korisničko ime ne postoji!"
							\end{packed_enum}
							\item[5.b] Pogrešna lozinka
							\begin{packed_enum}
								\item Prikazuje se poruka "Kriva lozinka!"
							\end{packed_enum}
						\end{packed_item}
					\end{packed_item}


				
					\noindent \underbar{\textbf{UC3 - promjena korisničkih podataka}}
					\begin{packed_item}
						\item \textbf{Glavni sudionik: registrirani korisnik, administrator}
						\item  \textbf{Cilj: promjena korisničkog imena i/ili lozinke}
						\item  \textbf{Sudionici: -}
						\item  \textbf{Preduvjet: korisnik je prijavljen u sustav}
						
						\item  \textbf{Opis osnovnog tijeka:}
						\item[] \begin{packed_enum}
							\item Korisnik kliknom na gumb pristupi postavkama
							\item Korisnik promijeni korisničko ime ili lozinku upisivanjem novog imena ili lozinke
							\item Korisnik klikne na gumb za spremanje promjena
							\item Sustavu se šalje zahtjev za promjenom podataka 
							\item U bazu podataka se upisuju promjene
							\item Korisnik se automatski odjavi za stranice
						\end{packed_enum}
						
						\item  \textbf{Opis mogućih odstupanja:}
						\item[] \begin{packed_item}
							\item[5.a] Novo korisničko ime je već zauzeto
							\begin{packed_enum}
								\item Prikazuje se poruka "Korisničko ime zauzeto!"
							\end{packed_enum}
							\item[5.b] Lozinka je slaba (sadrži manje od 6 znakova i/ili sadrži samo slova)
							\begin{packed_enum}
								\item Prikazuje se poruka "Lozinka mora imati najmanje 6 znakova te barem jedan broj ili specijalan znak!"
							\end{packed_enum}
						\end{packed_item}
					\end{packed_item}
				
				
				
					\noindent \underbar{\textbf{UC4 - brisanje korisničkog računa}}
					\begin{packed_item}
						\item \textbf{Glavni sudionik: registrirani korisnik, administrator}
						\item  \textbf{Cilj: brisanje korisničkog računa i svih podataka vezanih uz račun}
						\item  \textbf{Sudionici: -}
						\item  \textbf{Preduvjet: korisnik je prijavljen u sustav}
						
						\item  \textbf{Opis osnovnog tijeka:}
						\item[] \begin{packed_enum}
							\item Korisnik klikne na gumb za pristup postavkama
							\item Korisnik klikne na gumb za brisanje računa
							\item Pojavljuje se upozorenje i prozor sa zahtjevom za potvrdom
							\item Klikom na gumb "Da" šalje se zahtjev za brisanje računa sustavu
							\item Sustav briše račun iz baze podataka
							\item Korisnik se preusmjerava na homepage, te više nije prijavljen u sustav
						\end{packed_enum}
						
						\item  \textbf{Opis mogućih odstupanja:}
						\item[] \begin{packed_item}							
							\item[4.a] Korisnik klikne na gumb "Ne"
							\begin{packed_enum}
								\item Prozor sa zahtjevom se zatvara te korisnik ostaje u postavkama
							\end{packed_enum}
							\item[6.a] Brisanje korisničkog računa je obavio administrator
							\begin{packed_enum}
								\item Nakon brisanja korisničkog računa administrator ostaje u postavkama
							\end{packed_enum}
						\end{packed_item}
					\end{packed_item}



					\noindent \underbar{\textbf{UC5 - pregled liste recepata}}
					\begin{packed_item}
						\item \textbf{Glavni sudionik: neregistrirani korisnik, registrirani korisnik administrator}
						\item  \textbf{Cilj: prikazati recepte na web stranici}
						\item  \textbf{Sudionici: -}
						\item  \textbf{Preduvjet: -}
						
						\item  \textbf{Opis osnovnog tijeka:}
						\item[] \begin{packed_enum}
							\item Korisnik odabere filtere i način sortiranja recepata
							\item Korisnik u polje za pretraživanje upiše tekst
							\item Šalje se zahtjev za dohvaćanje recepata sustavu
							\item Sustav dohvaća recepte iz baze podataka
							\item Korisniku se prikaže lista recepata
						\end{packed_enum}
						
						\item  \textbf{Opis mogućih odstupanja:}
						\item[] \begin{packed_item}
							\item[1.a] Odabran je pregled vlastitih recepata
							\begin{packed_enum}
								\item Korak 2 se preskače i zahtjev se šalje za receptima koje je korisnik objavio
							\end{packed_enum}
							\item[1.b] Odabran je pregled omiljenih recepata
							\begin{packed_enum}
								\item Korak 2 se preskače i zahtjev se šalje za receptima koje je korisnik označio omiljenima
							\end{packed_enum}
							\item [2.a] Nije upisan tekst po kojem se pretražuje
							\begin{packed_enum}
								\item Prikazuju se svi recepti
							\end{packed_enum}
						\end{packed_item}
					\end{packed_item}


				
					\noindent \underbar{\textbf{UC6 - stvaranje novog recepta}}
					\begin{packed_item}
						\item \textbf{Glavni sudionik: registrirani korisnik}
						\item  \textbf{Cilj: kreiranje novog recepta}
						\item  \textbf{Sudionici: -}
						\item  \textbf{Preduvjet: korisnik prijavljen u sustav}
						
						\item  \textbf{Opis osnovnog tijeka:}
						\item[] \begin{packed_enum}
							\item Korisnik klikne na gumb za početak stvaranja novog recepta
							\item Korisnik upisuje podatke o receptu kao što su postupak, slike, videi, kategorije itd.
							\item Klikom na gumb za objavu recepta šalje se zahtjev sustavu
							\item Sustav sprema recept u bazu podataka
							\item Recept postaje vidljiv drugim korisnicima
						\end{packed_enum}
						
						\item  \textbf{Opis mogućih odstupanja:}
						\item[] \begin{packed_item}
							\item[3.a] Recept nema naslov
							\begin{packed_enum}
								\item Prikazuje se poruka "Potrebno je upisati ime recepta"
							\end{packed_enum}
							\item[3.b] Nije upisana lista sastojaka
							\begin{packed_enum}
								\item Prikazuje se poruka "Potrebno je upisati listu sastojaka"
							\end{packed_enum}
							\item[3.c] Ne postoji ni jedan korak pripreme
							\begin{packed_enum}
								\item Prikazuje se poruka "Potrebno je upisati barem jedan korak pripreme"
							\end{packed_enum}
							\item[3.d] Nije uneseno vrijeme pripreme
							\begin{packed_enum}
								\item Prikazuje se poruka "Potrebno je upisati vrijeme pripreme"
							\end{packed_enum}
						\end{packed_item}
					\end{packed_item}



					\noindent \underbar{\textbf{UC7 - promjena podataka o receptu}}
					\begin{packed_item}
						\item \textbf{Glavni sudionik: registrirani korisnik, administrator}
						\item  \textbf{Cilj: promjena bilo kojeg podatka o receptu kojeg je korisnik napravio}
						\item  \textbf{Sudionici: -}
						\item  \textbf{Preduvjet: korisnik je prijavljen u sustav te je on autor recepta kojeg želi promijeniti ili je administrator}
						
						\item  \textbf{Opis osnovnog tijeka:}
						\item[] \begin{packed_enum}
							\item Korisnik klikne na recept
							\item Korisnik klikne na gumb za uređivanje recepta
							\item Preusmjerava se na poveznicu za promjenu recepta
							\item Korisnik promijeni podatke o receptu i klikne na gumb za spremanje
							\item Sustavu se šalje zahtjev sa novim podacima o receptu
							\item U bazu podataka se spremaju promjene
							\item Korisnika se preusmjerava na poveznicu na kojoj se može pregledati recept
						\end{packed_enum}
						
						\item  \textbf{Opis mogućih odstupanja:}
						\item[] \begin{packed_item}							
							\item[4.a] Korisnik klikne na gumb za brisanje recepta
							\begin{packed_enum}
								\item Pojavljuje se upozorenje i prozor sa zahtjevom za potvrdom
								\item Korak 5 se preskače
							\end{packed_enum}
							\item[5.a] Problem u novim podacima (referencirati se na odstupanja u UC6) 
						\end{packed_item}
					\end{packed_item}



					\noindent \underbar{\textbf{UC8 - prijava na recepte korisnika}}
					\begin{packed_item}
						\item \textbf{Glavni sudionik: registrirani korisnik}
						\item  \textbf{Cilj: prijava korisnika na recepte autora kako bi primio obavijesti kada taj korisnik napravi novi recept}
						\item  \textbf{Sudionici: -}
						\item  \textbf{Preduvjet: korisnik je prijavljen u sustav}
						
						\item  \textbf{Opis osnovnog tijeka:}
						\item[] \begin{packed_enum}
							\item Prijavljeni korisnik otvara recept nekog autora
							\item Klikom na gumb za praćenje se sustavu šalje zahtjev
							\item U bazu podataka se zapisuje da je korisnik prijavljen na recepte odabranog autora
							\item Recepti odabranog autora prikazuju se u listi obavijesti
						\end{packed_enum}
						
						\item  \textbf{Opis mogućih odstupanja:}
						\item[] \begin{packed_item}							
							\item[3.a] Korisnik je već prijavljen na recepte autora
							\begin{packed_enum}
								\item Klikom na gumb za praćenje autora se korisnik odjavljuje sa recepata tog autora
							\end{packed_enum}
						\end{packed_item}
					\end{packed_item}
				
				
				
					\noindent \underbar{\textbf{UC9 - pregled obavijesti}}
					\begin{packed_item}
						\item \textbf{Glavni sudionik: registrirani korisnik}
						\item  \textbf{Cilj: pregled recepata autora koje korisnik prati}
						\item  \textbf{Sudionici: -}
						\item  \textbf{Preduvjet: korisnik je prijavljen u sustav}
						
						\item  \textbf{Opis osnovnog tijeka:}
						\item[] \begin{packed_enum}
							\item Korisnik klikne na gumb za pregled obavijesti
							\item Sustavu se šalje zahtjev
							\item Iz baze podataka čitaju se obavijesti korisnika
							\item Prikazuju se recepti svih autora koje korisnik prati
						\end{packed_enum}
						
						\item  \textbf{Opis mogućih odstupanja: -}
						\item[] \begin{packed_item}							
						\end{packed_item}
					\end{packed_item}
				
				
				
					\noindent \underbar{\textbf{UC10 - spremanje/bookmarkanje recepta}}
					\begin{packed_item}
						\item \textbf{Glavni sudionik: registrirani korisnik}
						\item  \textbf{Cilj: spremanje recepata kako bi se kasnije mogli pregledati omiljeni recepti}
						\item  \textbf{Sudionici: -}
						\item  \textbf{Preduvjet: korisnik prijavljen u sustav}
						
						\item  \textbf{Opis osnovnog tijeka:}
						\item[] \begin{packed_enum}
							\item korisnik klikne na gumb za spremanje recepta
							\item Šalje se zahtjev sustavu
							\item U bazu podataka se upisuje recept kao omiljen korisniku
						\end{packed_enum}
						
						\item  \textbf{Opis mogućih odstupanja:}
						\item[] \begin{packed_item}
							\item[3.a] Recept je već spremljen kao omiljen
							\begin{packed_enum}
								\item Recept se miče is liste omiljenih recepata
							\end{packed_enum}
						\end{packed_item}
					\end{packed_item}
				
				
				
					\noindent \underbar{\textbf{UC11 - pregled specifičnog recepta}}
					\begin{packed_item}
						\item \textbf{Glavni sudionik: neregistrirani korisnik, registrirani korisnik}
						\item  \textbf{Cilj: prikaz svih podataka vezanih uz recept}
						\item  \textbf{Sudionici: -}
						\item  \textbf{Preduvjet: -}
						
						\item  \textbf{Opis osnovnog tijeka:}
						\item[] \begin{packed_enum}
							\item Korisnik klikne na recept
							\item Sustavu se šalje zahtjev za dohvaćanje podataka o receptu
							\item Iz baze podataka se čitaju podaci o receptu
							\item Korisniku se prikazuju dohvaćeni podaci
						\end{packed_enum}
						
						\item  \textbf{Opis mogućih odstupanja: -}
						\item[] \begin{packed_item}							
						\end{packed_item}
					\end{packed_item}
					
					
					
					\noindent \underbar{\textbf{UC12 - komentiranje na recept}}
					\begin{packed_item}
						\item \textbf{Glavni sudionik: registrirani korisnik}
						\item  \textbf{Cilj: ostavljanje komentara na nekom receptu}
						\item  \textbf{Sudionici: -}
						\item  \textbf{Preduvjet: korisnik je prijavljen u sustav te je otvorio stranicu recepta}
						
						\item  \textbf{Opis osnovnog tijeka:}
						\item[] \begin{packed_enum}
							\item Scrollanjem do kraja recepta se dolazi do dijela sa komentarima
							\item Upisuje se tekst u polje za komentar
							\item Klikom na gumb za objavu komentara šalje se zahtjev sustavu
							\item U bazu podataka se upisuje komentar
							\item Web stranica se osvježava i komentar je sad vidljiv
						\end{packed_enum}
						
						\item  \textbf{Opis mogućih odstupanja:}
						\item[] \begin{packed_item}
							\item[3.a] Tekst komentara je prazan
							\begin{packed_enum}
								\item Gumb za objavljivanje komentara je blokiran te nije moguće kliknuti na njega
							\end{packed_enum}
						\end{packed_item}
					\end{packed_item}
				

					\noindent \underbar{\textbf{UC13 - razmjena poruka među registriranim korisnicima}}
					\begin{packed_item}
						\item \textbf{Glavni sudionik: registrirani korisnik, online-chat platforma}
						\item  \textbf{Cilj: razmjena poruka među korisnicima}
						\item  \textbf{Sudionici: -}
						\item  \textbf{Preduvjet: korisnik je prijavljen u sustav i omogućio je platformi pristup za komunikaciju}
						
						\item  \textbf{Opis osnovnog tijeka:}
						\item[] \begin{packed_enum}
							\item Korisnik na javnom profilu drugog korisnika klikne da gumb za uspostavu kontakta
							\item Sustav preusmjerava zahtjev online-chat platformi koja obrađuje daljnje zahtjeve
							\item Online-chat platforma prikazuje njihove dosadašnje poruke (ako ih ima)
							\item Korisnik upisuje novu poruku i stisne gumb za slanje
							\item Online-chat platforma poruku sprema i šalje drugom korisniku
							\item Korisniku se poruka prikazuje među dosad izmijenjenim porukama kao poslana
							\item Korisnik izlazi iz prozora za razmjenu poruka i poziva
							\item Online-chat platforma šalje sustavu obavijest o završetku radnje
							\item Sustav preuzima daljnju obradu zahtjeva
						\end{packed_enum}
						
						\item  \textbf{Opis mogućih odstupanja:}
							\begin{packed_item}
								\item[1.a] Korisnik kojeg se želi kontaktirati nije omogućio opciju komunikacije online-chat platformi
								\begin{packed_enum}
									\item Gumb za uspostavu kontakta je blokiran te nije moguće kliknuti na njega
									\item Svi daljnji koraci se ne izvode
								\end{packed_enum}
								\item[7.a] Korisnik želi poslati još poruka
								\begin{packed_enum}
									\item Ponavljaju se koraci 4, 5, i 6
								\end{packed_enum}
							\end{packed_item}
					\end{packed_item}
				


					\noindent \underbar{\textbf{UC14 - razmjena poziva među registriranim korisnicima}}
					\begin{packed_item}
						\item \textbf{Glavni sudionik: registrirani korisnici}
						\item  \textbf{Cilj: razmjena poziva među korisnicima}
						\item  \textbf{Sudionici: online-chat platforma}
						\item  \textbf{Preduvjet: korisnik je prijavljen u sustav i odobrio je platformi pristup mikrofonu i kameri}
						
						\item  \textbf{Opis osnovnog tijeka:}
						\item[] \begin{packed_enum}
							\item Korisnik na javnom profilu drugog korisnika klikne da gumb za uspostavu kontakta
							\item Sustav preusmjerava zahtjev online-chat platformi koja obrađuje daljnje zahtjeve
							\item Online-chat platforma prikazuje stare poruke i gumb za opciju pozivanja
							\item Korisnik klikne na gumb za opciju pozivanja
							\item Online-chat platforma dohvaća podatke o dostupnosti korisnika i prikazuje ih korisniku koji inicira komunikaciju
							\item Korisnik klikne gumb za uspostavljanje poziva
							\item Online-chat platforma upućuje poziv korisniku
							\item Korisnik kojeg se zove prihvaća poziv
							\item Klikom na gumb za prekid poziva šalje se zahtjev online-chat platformi koja gasi poziv
							\item Korisnik izlazi iz prozora za razmjenu poruka i poziva
							\item Online-chat platforma šalje sustavu obavijest o završetku radnje
							\item Sustav preuzima daljnju obradu zahtjeva
						\end{packed_enum}
						
						\item  \textbf{Opis mogućih odstupanja:}
							\begin{packed_item}
								\item[1.a] Korisnik kojeg se želi kontaktirati nije omogućio opciju komunikacije online-chat platformi
								\begin{packed_enum}
									\item Gumb za uspostavu kontakta je blokiran te nije moguće kliknuti na njega
									\item Svi daljnji koraci se ne izvode
								\end{packed_enum}
								\item[6.a] Korisnik kojeg se pokušava nazvati nije dostupan u tom vremenu
								\begin{packed_enum}
									\item Gumb za uspostavu poziva je blokiran te nije moguće kliknuti na njega
									\item Prikazuje se kalendar u kojem su označeni termini u kojima je traženi korisnik dostupan
									\item Korisnik inicijator može označiti termin u kojem želi razgovarati
									\item Online-chat platforma zapisuje taj podatak i prenosi ga traženom korisniku
									\item Koraci 7, 8 i 9 se ne izvode
								\end{packed_enum}
								\item[8.a] Korisnik kojeg se zove odbija poziv
								\begin{packed_enum}
									\item Poziv se prekida
								\end{packed_enum}
							\end{packed_item}
					\end{packed_item}



					\noindent \underbar{\textbf{UC15 - označavanje dostupnosti za pozive}}
					\begin{packed_item}
						\item \textbf{Glavni sudionik: registrirani korisnik}
						\item  \textbf{Cilj: spremanje termina slobodnih za pozive i njihov prikaz drugim korisnicima}
						\item  \textbf{Sudionici: online-chat platforma}
						\item  \textbf{Preduvjet: korisnik je prijavljen u sustav i omogućio je platformi pristup za komunikaciju}
						
						\item  \textbf{Opis osnovnog tijeka:}
						\item[] \begin{packed_enum}
							\item Korisnik klikne na gumb za postavke
							\item Korisnik klikne na gumb za uređivanje kalendara
							\item Šalje se zahtjev sustavu za dohvaćanje dosad unesenih podataka u kalendaru
							\item Šalje se zahtjev online-chat platformi za podatke o zatraženim pozivima
							\item Korisniku se prikazuje kalendar s njegovim označenim terminima i zahtjevima za poziv
							\item Korisnik klikom na određeni dan upisuje vrijeme kad je slobodan i stisne gumb za spremanje
							\item Sustav pohranjuje promjene kalendara u bazu podataka
							\item Sustav šalje platformi novouneseni podatak
							\item Korisnik izlazi iz postavki
						\end{packed_enum}
						
						\item  \textbf{Opis mogućih odstupanja:}
							\begin{packed_item}
								\item[6.a] Korisnik klikne na termin zatraženog poziva
								\begin{packed_enum}
									\item Pojavljuje se prozor za prihvaćanje odnosno odbijanje zahtjeva
									\item Korisnik bira klikom na gumb za prihvaćanje odnosno odbijanje
								\end{packed_enum}
							\end{packed_item}
					\end{packed_item}

						% \noindent \underbar{\textbf{UC}}
					% \begin{packed_item}
					% 	\item \textbf{Glavni sudionik: }
					% 	\item  \textbf{Cilj:}
					% 	\item  \textbf{Sudionici:}
					% 	\item  \textbf{Preduvjet:}
						
					% 	\item  \textbf{Opis osnovnog tijeka:}
					% 	\item[] \begin{packed_enum}
					% 	\end{packed_enum}
						
					% 	\item  \textbf{Opis mogućih odstupanja:}
					% 	\item[] \begin{packed_item}							
					% 	\end{packed_item}
					% \end{packed_item}	
					
					\eject
					\noindent \textbf{Dijagrami obrazaca uporabe}
					
					\begin{figure}[H]
						\centering
						\includegraphics[width=1.2\linewidth]{"slike/dijagrami/UML cijeli"}
						\caption{Cijeli UML dijagram}
						\label{fig:uml-cijeli}
					\end{figure}
				
					\begin{figure}[H]
						\centering
						\includegraphics[width=1.2\linewidth]{"slike/dijagrami/UML 1-4"}
						\caption{UC1-UC4}
						\label{fig:uml-1-4}
					\end{figure}
				
					\begin{figure}[H]
						\centering
						\includegraphics[width=1.2\linewidth]{"slike/dijagrami/UML 13-15"}
						\caption{UC13-UC15}
						\label{fig:uml-13-15}
					\end{figure}
		
				
				\eject		
				
			\subsection{Sekvencijski dijagrami}
				
				\begin{figure}[H]
					\centering
					\includegraphics[width=1.1\linewidth]{"slike/dijagrami/Seq 1-4"}
					\caption{Seq 1-4}
					\label{fig:seq-1-4}
				\end{figure}
				
				\begin{figure}[H]
					\centering
					\includegraphics[width=1.2\linewidth]{"slike/dijagrami/Seq 6-7"}
					\caption{Seq 6-7}
					\label{fig:seq-6-7}
				\end{figure}
				
				\begin{figure}[H]
					\centering
					\includegraphics[width=1.2\linewidth]{"slike/dijagrami/Seq 8-10,12"}
					\caption{Seq 8-10;12}
					\label{fig:seq-8-10;12}
				\end{figure}
				
				\begin{figure}[H]
					\centering
					\includegraphics[width=1.2\linewidth]{"slike/dijagrami/Seq 13-14"}
					\caption{Seq 13-14}
					\label{fig:seq-13-14}
				\end{figure}
				
				\eject
	
		\section{Ostali zahtjevi}
		
			\textbf{\textit{dio 1. revizije}}\\
		 
			 \textit{Nefunkcionalni zahtjevi i zahtjevi domene primjene dopunjuju funkcionalne zahtjeve. Oni opisuju \textbf{kako se sustav treba ponašati} i koja \textbf{ograničenja} treba poštivati (performanse, korisničko iskustvo, pouzdanost, standardi kvalitete, sigurnost...). Primjeri takvih zahtjeva u Vašem projektu mogu biti: podržani jezici korisničkog sučelja, vrijeme odziva, najveći mogući podržani broj korisnika, podržane web/mobilne platforme, razina zaštite (protokoli komunikacije, kriptiranje...)... Svaki takav zahtjev potrebno je navesti u jednoj ili dvije rečenice.}
			 
			 
			 
	