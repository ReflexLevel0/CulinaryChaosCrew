\chapter{Specifikacija programske potpore}
		
	\section{Funkcionalni zahtjevi}
			
			\textbf{\textit{dio 1. revizije}}\\
			
			\textit{Navesti \textbf{dionike} koji imaju \textbf{interes u ovom sustavu} ili  \textbf{su nositelji odgovornosti}. To su prije svega korisnici, ali i administratori sustava, naručitelji, razvojni tim.}\\
				
			\textit{Navesti \textbf{aktore} koji izravno \textbf{koriste} ili \textbf{komuniciraju sa sustavom}. Oni mogu imati inicijatorsku ulogu, tj. započinju određene procese u sustavu ili samo sudioničku ulogu, tj. obavljaju određeni posao. Za svakog aktora navesti funkcionalne zahtjeve koji se na njega odnose.}\\
			
			
			\noindent \textbf{Dionici:}
			
			\begin{packed_enum}
				
				\item Korisnik
				\item Administrator				
				\item Razvojni tim
				
			\end{packed_enum}
			
			\noindent \textbf{Aktori i njihovi funkcionalni zahtjevi:}
			
			
			\begin{packed_enum}
				\item  \underbar{Neregistrirani korisnik (inicijator) može: }
				\begin{packed_enum}
					\item vidjeti listu recepata svih korisnika
					\begin{packed_enum}
						\item filtrirati listu
						\item sortirati listu
					\end{packed_enum}
					\item pretraživati recepte po naslovu i korisniku
					\item napraviti novi račun
				\end{packed_enum}
			
				\item  \underbar{Registrirani korisnik (inicijator) može:}
				\begin{packed_enum}
					\item raditi nove recepte
					\item pregledati listu svojih recepata
					\item obrisati vlastiti recept
					\item promijeniti detalje računa:
					\begin{packed_enum}
						\item korisničko ime
						\item lozinku
					\end{packed_enum}
					\item deaktivirati račun
				\end{packed_enum}
			
				\item\underbar{Administrator (inicijator) može:}
				\begin{packed_enum}
					\item pregledati korisničke podatke
					\item pregledati recepte
					\item promijeniti korisničke podatke
					\item promijeniti podatke o receptu
					\item izbrisati korisnički račun
					\item izbrisati recept
				\end{packed_enum}
			
				\item\underbar{Baza podataka (sudionik):}
				\begin{packed_enum}
					\item pohranjuje podatke o:
					\begin{packed_enum}
						\item korisnicima
						\item receptima
					\end{packed_enum}
				\end{packed_enum}
			
				\item\underbar{Web stranica(sudionik):}
				\begin{packed_enum}
					\item omogućava korisnicima interakciju sa:
					\begin{packed_enum}
						\item drugim korisnicima
						\item listom recepata
					\end{packed_enum}
				\end{packed_enum}
			
			\end{packed_enum}
			
			\eject 
			
			
				
			\subsection{Obrasci uporabe}
				
				\textbf{\textit{dio 1. revizije}}
				
				\subsubsection{Opis obrazaca uporabe}
					\textit{Funkcionalne zahtjeve razraditi u obliku obrazaca uporabe. Svaki obrazac je potrebno razraditi prema donjem predlošku. Ukoliko u nekom koraku može doći do odstupanja, potrebno je to odstupanje opisati i po mogućnosti ponuditi rješenje kojim bi se tijek obrasca vratio na osnovni tijek.}\\



					\noindent \underbar{\textbf{UC01 - stvaranje korisničkog računa}}
					\begin{packed_item}
						\item \textbf{Glavni sudionik: korisnik}
						\item  \textbf{Cilj: stvaranje novog korisničkog računa}
						\item  \textbf{Sudionici: web stranica, baza podataka}
						\item  \textbf{Preduvjet: -}
						
						\item  \textbf{Opis osnovnog tijeka:}
						\item[] \begin{packed_enum}
							\item korisnik klikne na gumb za prijavu
							\item korisnik upiše  željeno korisničko ime i lozinku
							\item korisnik klikne na gumb "Registracija"
							\item web stranica se preusmjeri na homepage, ali sa korisnikom prijavljenim u sustav
						\end{packed_enum}
						
						\item  \textbf{Opis mogućih odstupanja:}
						\item[] \begin{packed_item}							
							\item[2.a] korisničko ime je zauzeto (već postoji)
							\begin{packed_item}
								\item prikazuje se poruka "Korisničko ime zauzeto!"
							\end{packed_item}
							\item[2.b] lozinka je slaba (sadrži manje od 6 znakova i/ili sadrži samo slova)
							\begin{packed_item}
								\item prikazuje se poruka "Lozinka mora imati najmanje 6 znakova te barem jedan broj ili specijalan znak!"
							\end{packed_item}
						\end{packed_item}
					\end{packed_item}


				
					\noindent \underbar{\textbf{UC02 - prijava u korisnički račun}}
					\begin{packed_item}
						\item \textbf{Glavni sudionik: korisnik}
						\item  \textbf{Cilj: prijava korisnika u vlastiti račun}
						\item  \textbf{Sudionici: web stranica, baza podataka}
						\item  \textbf{Preduvjet: -}
						
						\item  \textbf{Opis osnovnog tijeka:}
						\item[] \begin{packed_enum}
							\item korisnik klikne na gumb za prijavu
							\item korisnik unosi korisničko ime i lozinku
							\item korisnik klikne na gumb "Prijava"
							\item web stranica se preusmjeri na homepage sa korisnikom koji je sada prijavljen
						\end{packed_enum}
						
						\item  \textbf{Opis mogućih odstupanja:}
						\item[] \begin{packed_item}
							\item[2.a] korisničko ime ne postoji u bazi podataka
							\begin{packed_item}
								\item prikazuje se poruka "Korisničko ime ne postoji!"
							\end{packed_item}
							\item[2.b] pogrešna lozinka
							\begin{packed_item}
								\item prikazuje se poruka "Kriva lozinka!"
							\end{packed_item}
						\end{packed_item}
					\end{packed_item}


				
					\noindent \underbar{\textbf{UC03 - promjena korisničkih podataka}}
					\begin{packed_item}
						\item \textbf{Glavni sudionik: korisnik}
						\item  \textbf{Cilj: promjena korisničkog imena i/ili lozinke}
						\item  \textbf{Sudionici: web stanica, baza podataka}
						\item  \textbf{Preduvjet: korisnik je prijavljen u sustav}
						
						\item  \textbf{Opis osnovnog tijeka:}
						\item[] \begin{packed_enum}
							\item korisnik klikne na gumb "Postavke"
							\item[2.a] korisnik promijeni korisničko ime upisivanjem novog imena
							\item[2.b] korisnik promijeni lozinku upisivanjem nove lozinke 
							\item klikom na gumb "Pohrani promjene" se u bazu podataka upisuju promjene
						\end{packed_enum}
						
						\item  \textbf{Opis mogućih odstupanja:}
						\item[] \begin{packed_item}
							\item[3.a] novo korisničko ime je zauzeto
							\begin{packed_item}
								\item prikacuje se poruka "Korisničko ime zauzeto!"
							\end{packed_item}
							\item[3.b] lozinka je slaba (sadrži manje od 6 znakova i/ili sadrži samo slova)
							\begin{packed_item}
								\item prikazuje se poruka "Lozinka mora imati najmanje 6 znakova te barem jedan broj ili specijalan znak!"
							\end{packed_item}
						\end{packed_item}
					\end{packed_item}
				
				
				
					\noindent \underbar{\textbf{UC04 - brisanje korisničkog računa}}
					\begin{packed_item}
						\item \textbf{Glavni sudionik: korisnik}
						\item  \textbf{Cilj: brisanje korisničkog računa i svih podataka vezanih uz račun}
						\item  \textbf{Sudionici: web stanica, baza podataka}
						\item  \textbf{Preduvjet: korisnik je prijavljen u sustav}
						
						\item  \textbf{Opis osnovnog tijeka:}
						\item[] \begin{packed_enum}
							\item korisnik klikne na gumb "Postavke"
							\item korisnik klikne na gumb "Obriši račun"
							\item pojavljuje se obavijest "Ovim postupkom će se obrisati svi Vaši receptii vaš račun, jesti li sigurni da želite nastaviti?"
							\item klikom na gumb "Da" se briše račun i svi podaci vezani za račun iz baze podataka
							\item korisnik se preusmjerava na homepage, te više nije prijavljen u sustav
						\end{packed_enum}
						
						\item  \textbf{Opis mogućih odstupanja:}
						\item[] \begin{packed_item}							
							\item[2.a] korisnik kliknuo na gumb "Ne"
							\begin{packed_item}
								\item obavijest se zatvara te korisnik ostaje u postavkama
							\end{packed_item}
						\end{packed_item}
					\end{packed_item}



					\noindent \underbar{\textbf{UC05 - pregled liste recepata}}
					\begin{packed_item}
						\item \textbf{Glavni sudionik: korisnik}
						\item  \textbf{Cilj: prikazati recepte na web stranici}
						\item  \textbf{Sudionici: web stranica, baza podataka}
						\item  \textbf{Preduvjet: -}
						
						\item  \textbf{Opis osnovnog tijeka:}
						\item[] \begin{packed_enum}
							\item korisnik odabere filtere i način sortiranja recepata
							\item korisnik u polje za pretraživanje upiše tekst
							\item web stranica poziva API te dohvaća i prikazuje listu recepata
						\end{packed_enum}
						
						\item  \textbf{Opis mogućih odstupanja:}
						\item[] \begin{packed_item}
							\item [2.a] nije upisan tekst po kojem se pretražuje
							\item[] \begin{packed_item}
								\item prikazuju se svi recepti
							\end{packed_item}
						\end{packed_item}
					\end{packed_item}


				
					\noindent \underbar{\textbf{UC06 - stvaranje novog recepta}}
					\begin{packed_item}
						\item \textbf{Glavni sudionik: korisnik}
						\item  \textbf{Cilj: kreiranje novog recepta}
						\item  \textbf{Sudionici: web stranica, baza podataka}
						\item  \textbf{Preduvjet: korisnik prijavljen u sustav}
						
						\item  \textbf{Opis osnovnog tijeka:}
						\item[] \begin{packed_enum}
							\item korisnik klikne na gumb "Novi recept"
							\item upišu se podaci o receptu kao postupak, slike, videi, kategorije itd.
							\item klikom na objavi recept se on sprema u bazu podataka
							\item recept postaje vidljiv drugim korisnicima
						\end{packed_enum}
						
						\item  \textbf{Opis mogućih odstupanja:}
						\item[] \begin{packed_item}
							\item[2.a] recept nema naslov
							\begin{packed_item}
								\item prikazuje se poruka "Potrebno je upisati ime recepta"
							\end{packed_item}
							\item[2.b] nije upisana lista sastojaka
							\begin{packed_item}
								\item prikazuje se poruka "Potrebno je upisati listu sastojaka"
							\end{packed_item}
							\item[2.c] ne postoji ni jedan korak pripreme
							\begin{packed_item}
								\item prikazuje se poruka "Potrebno je upisati barem jedan korak pripreme"
							\end{packed_item}
							\item[2.d] nije uneseno vijeme kuhanja
							\begin{packed_item}
								\item prikazuje se poruka "Potrebno je upisati vrijeme kuhanja"
							\end{packed_item}
						\end{packed_item}
					\end{packed_item}



					\noindent \underbar{\textbf{UC07 - prijava na recepte korisnika}}
					\begin{packed_item}
						\item \textbf{Glavni sudionik: korisnika}
						\item  \textbf{Cilj: prijava korisnika na recepte autora kako bi primio obavijesti kada taj korisnik napravi novi recept}
						\item  \textbf{Sudionici: web stranica, baza podataka}
						\item  \textbf{Preduvjet: korisnik je prijavljen u sustav}
						
						\item  \textbf{Opis osnovnog tijeka:}
						\item[] \begin{packed_enum}
							\item prijavljeni korisnik otvara recept nekog autora
							\item klikom na gumb "???" se korisnik prijavljuje na recepte autora recepta
							\item recepti autora se pokazuju u listi obavijesti
						\end{packed_enum}
						
						\item  \textbf{Opis mogućih odstupanja:}
						\item[] \begin{packed_item}							
							\item korisnik je već prijavljen na recepte autora
							\begin{packed_item}
								\item klikom na gumb za prijavu na recepte autora se korisnik odjavljuje sa recepata tog autora
							\end{packed_item}
						\end{packed_item}
					\end{packed_item}
				
				
				
					\noindent \underbar{\textbf{UC08 - pregled obavijesti}}
					\begin{packed_item}
						\item \textbf{Glavni sudionik: korisnik}
						\item  \textbf{Cilj: pregled recepata autora koje korisnik prati}
						\item  \textbf{Sudionici: web stanica, baza podataka}
						\item  \textbf{Preduvjet: korisnik je prijavljen u sustav}
						
						\item  \textbf{Opis osnovnog tijeka:}
						\item[] \begin{packed_enum}
							\item korisnik klikne na gumb koji izgleda kao zvonce
							\item prikazuju se recepti svih autora koje korisnik prati
						\end{packed_enum}
						
						\item  \textbf{Opis mogućih odstupanja:}
						\item[] \begin{packed_item}							
						\end{packed_item}
					\end{packed_item}
				
				
				
					\noindent \underbar{\textbf{UC09 - spremanje/bookmarkanje recepta}}
					\begin{packed_item}
						\item \textbf{Glavni sudionik: korisnik}
						\item  \textbf{Cilj: spremanje recepata kako bi se kasnije mogli pregledati omiljeni recepti}
						\item  \textbf{Sudionici: web stanica, baza podataka}
						\item  \textbf{Preduvjet: korisnik prijavljen u sustav}
						
						\item  \textbf{Opis osnovnog tijeka:}
						\item[] \begin{packed_enum}
							\item[1.a] korisnik klikne na gumb za spremanje recepta iz liste recepata
							\item[1.b] korisnik klikne na gumb za spremanje recepata prilikom prikaza pojedinog recepta
							\item[2.] recept se sprema pod omiljene recepte
						\end{packed_enum}
						
						\item  \textbf{Opis mogućih odstupanja:}
						\item[] \begin{packed_item}
							\item recept je već spremljen kao omiljen
							\begin{packed_item}
								\item recept se miče is liste omiljenih recepata
							\end{packed_item}
						\end{packed_item}
					\end{packed_item}
				
				
				
					\noindent \underbar{\textbf{UC10 - pregled specifičnog recepta}}
					\begin{packed_item}
						\item \textbf{Glavni sudionik: korisnik}
						\item  \textbf{Cilj: prikaz svih podataka vezanih uz recept}
						\item  \textbf{Sudionici: web stranica, baza podataka}
						\item  \textbf{Preduvjet: -}
						
						\item  \textbf{Opis osnovnog tijeka:}
						\item[] \begin{packed_enum}
							\item korisnik kliknuo na recept
						\end{packed_enum}
						
						\item  \textbf{Opis mogućih odstupanja:}
						\item[] \begin{packed_item}							
						\end{packed_item}
					\end{packed_item}
					
					
					
					\noindent \underbar{\textbf{UC11 - komentiranje na recept}}
					\begin{packed_item}
						\item \textbf{Glavni sudionik: korisnik}
						\item  \textbf{Cilj: ostavlanje komentara na nekom receptu}
						\item  \textbf{Sudionici: web stranica, baza podataka}
						\item  \textbf{Preduvjet: korisnik je prijavljen u sustav te je otvorio stranicu recepta}
						
						\item  \textbf{Opis osnovnog tijeka:}
						\item[] \begin{packed_enum}
							\item scrollanjem do kraja recepta se dolazi do dijela sa komentarima
							\item upisuje se tekst u polje za komentar
							\item klikom na gumb "Objavi komentar" se on:
							\begin{packed_enum}
								\item upisuje u bazu podataka
								\item odmah prikazuje na korisnikovom ekranu
							\end{packed_enum}
						\end{packed_enum}
						
						\item  \textbf{Opis mogućih odstupanja:}
						\item[] \begin{packed_item}
							\item[2.a] tekst komentara je prazno
							\begin{packed_enum}
								\item gumb za objavljivanje komentara je zasivljen te je nemoguće kliknuti na njega
							\end{packed_enum}
						\end{packed_item}
					\end{packed_item}
				
				
				
					\noindent \underbar{\textbf{UC}}
					\begin{packed_item}
						\item \textbf{Glavni sudionik: }
						\item  \textbf{Cilj:}
						\item  \textbf{Sudionici:}
						\item  \textbf{Preduvjet:}
						
						\item  \textbf{Opis osnovnog tijeka:}
						\item[] \begin{packed_enum}
						\end{packed_enum}
						
						\item  \textbf{Opis mogućih odstupanja:}
						\item[] \begin{packed_item}							
						\end{packed_item}
					\end{packed_item}
					
					
					
				\subsubsection{Dijagrami obrazaca uporabe}
					
					\textit{Prikazati odnos aktora i obrazaca uporabe odgovarajućim UML dijagramom. Nije nužno nacrtati sve na jednom dijagramu. Modelirati po razinama apstrakcije i skupovima srodnih funkcionalnosti.}
				\eject		
				
			\subsection{Sekvencijski dijagrami}
				
				\textbf{\textit{dio 1. revizije}}\\
				
				\textit{Nacrtati sekvencijske dijagrame koji modeliraju najvažnije dijelove sustava (max. 4 dijagrama). Ukoliko postoji nedoumica oko odabira, razjasniti s asistentom. Uz svaki dijagram napisati detaljni opis dijagrama.}
				\eject
	
		\section{Ostali zahtjevi}
		
			\textbf{\textit{dio 1. revizije}}\\
		 
			 \textit{Nefunkcionalni zahtjevi i zahtjevi domene primjene dopunjuju funkcionalne zahtjeve. Oni opisuju \textbf{kako se sustav treba ponašati} i koja \textbf{ograničenja} treba poštivati (performanse, korisničko iskustvo, pouzdanost, standardi kvalitete, sigurnost...). Primjeri takvih zahtjeva u Vašem projektu mogu biti: podržani jezici korisničkog sučelja, vrijeme odziva, najveći mogući podržani broj korisnika, podržane web/mobilne platforme, razina zaštite (protokoli komunikacije, kriptiranje...)... Svaki takav zahtjev potrebno je navesti u jednoj ili dvije rečenice.}
			 
			 
			 
	