\chapter{Zaključak i budući rad}
		
		% \textbf{\textit{dio 2. revizije}}\\
		
		% %  \textit{U ovom poglavlju potrebno je napisati osvrt na vrijeme izrade projektnog zadatka, koji su tehnički izazovi prepoznati, jesu li riješeni ili kako bi mogli biti riješeni, koja su znanja stečena pri izradi projekta, koja bi znanja bila posebno potrebna za brže i kvalitetnije ostvarenje projekta i koje bi bile perspektive za nastavak rada u projektnoj grupi.}
		% 
		\noindent Izuzetno je mnogo vremena bilo potrošeno kako bi nadoknadili manjak jednog člana. Sve tehnologije koje smo koristili su nam bile nepoznate te smo zato čvsto odredili uloge kako bi smanjili obujam učenja pojedinog člana.
		 \noindent Backend je morao savladati Spring Boot i Maven sa svim svojim značajkama. Frontend je morao ovladati React-om za uređivanje komponenti stranice i npm-om za instalaciju paketa. Ljudi zaduženi za dokumentaciju su morali ovladati LaTex-om.
		 \noindent No kako to obično biva u malim timovima na kraju je svatko radio sve. Kada bi isti tim ljudi duže programirao zajedno te se dogovorio oko normi oblikovanja mogli bi biti brži i efektivniji, no to je bilo izuzetno teško sa našim popunjenim rasporedima.
		 \noindent Perspektiva koju smo svi usvojili je značajnost rada svakog od nas. Koliko god se pomoć činila mala na kraju bi njena značajnost imala veliku težinu.
		 \noindent \textbf{Ukratko, najbitnija je kontinuiranost u radu jer projekt iako kratak nije gotov u jednom danu.}
		 
		% \textit{Potrebno je točno popisati funkcionalnosti koje nisu implementirane u ostvarenoj aplikaciji.}
		 \noindent Unatoč volji, perspektivi te požrtvovnosti svih članova tima sljedeće značajke nismo uspijeli implementirati:
		\begin{itemize}
			\item video-chat između korisnika
			\item admin page (sa svojim značajkama)
			\item praćenje korisnika
			\item mijenjanje recepata
		\end{itemize}
		
		\eject 